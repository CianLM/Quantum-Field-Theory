\lecture{5}{21/10/2024}{Vacuum Energy}

Observe that while the Hamiltonian for the free massive scalar field can be written as
\begin{align}
    H = \int \dd{^3x} \mathcal{H} = \frac{1}{2} \int \dd{^3x} \left( \pi^2 + \left( \grad \phi \right)^2 + m^2 \phi^2 \right) 
,\end{align}
we desire an expression in terms of $a$ and $a^{\dag}$. Expanding the fields in terms of these operators, we see
\begin{align}
    H &= \frac{1}{2} \int \dd{^3x} \int \frac{\dd{^3p} \dd{^3q}}{\left( 2\pi \right)^{6}}  \bigg(-\frac{\sqrt{\omega_{\vb{p}} \omega_{\vb{q}}} }{2} \left( a_{\vb{p}} e^{i\vb{p}\cdot \vb{x}} - a^{\dag}_{\vb{p}} e^{-i \vb{p} \cdot \vb{x}} \right) \left( a_{\vb{q}} e^{i\vb{q}\cdot \vb{x}} - a^{\dag}_{\vb{q}} e^{-i \vb{q} \cdot \vb{x}} \right) \nonumber \\
    &\quad ~ -\frac{1}{2} \frac{1}{\sqrt{\omega_{\vb{p}} \omega_{\vb{q}}} } \left( a_{\vb{p}} e^{-i \vb{p} \cdot \vb{x}} - a_{\vb{p}}^{\dag}e^{i \vb{p} \cdot \vb{x}} \right)\left( a_{\vb{q}} e^{-i \vb{q} \cdot \vb{x}} - a^{\dag}_{\vb{q}}e^{i \vb{q} \cdot \vb{x}} \right) \vec{p} \vec{q} \nonumber\\
    & \quad~+ \frac{m^2}{2} \frac{1}{\sqrt{\omega_{\vb{p}} \omega_{\vb{q}}} } \left( a_{\vb{p}} e^{-i \vb{p} \cdot \vb{x}} - a^{\dag}_{\vb{p}}e^{i \vb{p} \cdot \vb{x}} \right)\left( a_{\vb{q}} e^{-i \vb{q} \cdot \vb{x}} - a^{\dag}_{\vb{q}}e^{i \vb{q} \cdot \vb{x}} \right) \bigg)\\
    &= \frac{1}{2} \int \frac{\dd{^3p}}{\left( 2\pi \right)^3} \frac{1}{2 \omega_{\vb{p}}} \bigg[ \underbrace{\left( - \omega_{\vb{p}}^2 +  \vec{p}^2 + m^2 \right)}_{\text{e.o.m. thus vanishes}} \left( a_{\vb{p}} a_{-\vb{p}} e^{-2i \omega t} + a_{\vb{p}}^{\dag} a_{-\vb{p}}^{\dag} e^{2i \omega t} \right)\\
        &\quad ~ + \left( \omega_{\vb{p}}^2 + \vec{p}^2 + m^2 \right) \left( a_{\vb{p}}^{\dag} a_{\vb{p}} + a_{\vb{p}} a_{\vb{p}}^{\dag} \right)  \bigg]  \\
        H &= \frac{1}{2} \int \frac{\dd{^3p}}{\left( 2\pi \right)^3} \omega_{\vb{p}} \left( a^{\dag}_{\vb{p}} a_{\vb{p}} + a_{\vb{p}} a^{\dag}_{\vb{p}} \right)  \\
        H &=  \int \frac{\dd{^3p}}{\left( 2\pi \right)^3} \omega_{\vb{p}} a^{\dag}_{\vb{p}} a_{\vb{p}} + \frac{1}{2}\int \frac{\dd{^3p}}{\left( 2\pi \right)^3} \omega \left( 2\pi \right)^3 \delta^{3}\left( 0 \right)  
.\end{align}

I have skipped some calculation steps, I \textit{highly recommend} one attempts to repeat this calculation and fill them in to make sure you understand what is being done. 

This last term is unusual, and appears unphysical as with a vacuum $\ket{0}$ satisfying $a_{\vb{p}} \ket{0} = 0$, we see
\begin{align}
    H \ket{0} = \frac{1}{2}\int \frac{\dd{^3p}}{\left( 2\pi \right)^3} \omega_{\vb{p}} \left( 2\pi \right)^3 \delta \left( 0 \right) \ket{0} = E_0 \ket{0} \to \infty
.\end{align}

To understand the nature of this, we need to see the origin of the divergence. There are in fact two divergences here:
\begin{itemize}
    \item An \textit{infrared divergence}: $\left( 2\pi \right)^3 \delta \left( 0 \right) $, associated with long distances, as it came from
        \begin{align}
            \delta \left( 0 \right) = \lim_{L \to \infty} \int_{-L}^{L} \dd{^3x} e^{- i\vec{x}\cdot \vec{p}} \bigg|_{\vec{p}=0} = \lim_{L \to \infty} \int_{-L}^{L} \dd{^3x} = V
        ,\end{align}
        a diverging volume $V$. As we are discussing an system with infinite size, we can instead discuss energy \textit{densities} (i.e. per unit volume) such that
        \begin{align}
            \epsilon_0 = \frac{E_0}{V} = \int \frac{\dd{^3p}}{\left( 2\pi \right)^3} \frac{1}{2}\omega_{\vb{p}} \sim  \int \dd{^3p} \vec{p}^2
        ,\end{align}
        which is still divergent. 
    \item Namely, it is an \textit{ultraviolet divergence}. Suppose one is performing
        \begin{align}
            \int_0^{\Lambda} \dd{^3p} \sqrt{\vec{p}^2 + m^2} \overset{\Lambda \to \infty}{\too} \infty
        ,\end{align}
        we see that this is a high frequency divergence. It is absurd to think that the theory is valid for arbitrarily high energies, and thus it is valid to consider a maximum energy scale of applicability, a cutoff, $\Lambda$.

        The solution here, is to declare that
        \begin{align}
            H \equiv \int \frac{\dd{^3p}}{\left( 2\pi \right)^3} \omega_{\vb{p}} a_{\vb{p}}^{\dag} a_{\vb{p}}
        .\end{align}

        One can convince themself that we can only measure energy differences and thus can remove this vacuum energy. However, practically, it is best to just take this $H$ as definition such that it fixes an ambiguity. There is an ambiguity in the \textit{normal ordering} of operators when one converts between classical and quantum field theories. Here it is clear that this $H$ is the correct definition in quantum field theory as it provides $H \ket{0} = 0$.
\end{itemize}

\begin{definition}
    If you have a list of fields, we define \textbf{normal ordering} as
    \begin{align}
        :\phi_1 \left( x_1 \right) \phi_2 \left( x_2 \right) \cdots \phi_{n}\left( x_{n} \right) :
    ,\end{align}
    where this is the usual product but we put creation operators $a^{\dag}_{\vb{p}}$ to the left of annihilation operators $a_{\vb{p}}$.
\end{definition}

\subsection{Fock Space}

We have the vacuum $\ket{0}$ and want to construct excited states atop it. It is usefully to observe that
\begin{align}
    \left[ H, a^{\dag}_{\vb{p}} \right] = \omega_{\vb{p}} a_{\vb{p}}^{\dag}
,\end{align}
and
\begin{align}
    \left[ H, a_{\vb{p}} \right] = - \omega_{\vb{p}} a_{\vb{p}}
.\end{align}

We then aim to construct energy eigenstates by
\begin{align}
    \ket{\vec{p}} = a_{\vb{p}} \ket{0}
.\end{align}
This is a single particle state. Observe that then
\begin{align}
    H \ket{\vec{p}} = \omega_{\vb{p}} \ket{\vec{p}}
.\end{align}

\begin{note}
    Until now I have noted operators $a^{\dag}_{\vb{p}}$ and $\omega_{\vb{p}}$ with $\vb{p}$, however they both only depend on the spatial component $\vec{p}$. I will now switch to this before dropping any decoration on $p$ at all when it is clear from context.
\end{note}
