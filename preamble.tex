% Some basic packages
%\usepackage[utf8]{inputenc}
\usepackage[T1]{fontenc}
\usepackage{textcomp}
\usepackage{url}
\usepackage{graphicx}
\usepackage{float}
\usepackage{booktabs}
\usepackage{enumitem}
\usepackage{siunitx}
\usepackage[hidelinks]{hyperref}
%\usepackage{authblk}

\usepackage{caption}
\usepackage{subcaption}

% lualatex setting
\usepackage[activate={true,nocompatibility},final,tracking=true,factor=1100,stretch=10,shrink=10]{microtype}

\usepackage{pdfcomment}
% pdftex settings
%\usepackage[activate={true,nocompatibility},final,tracking=true,kerning=true,spacing=true,factor=1100,stretch=10,shrink=10]{microtype}
%\DeclareMicrotypeSetDefault[spacing]{basictext}
%\microtypecontext{spacing=nonfrench}

\usepackage{geometry}

%\pdfminorversion=7

% Don't indent paragraphs, leave some space between them
\usepackage{parskip}

% Hide page number when page is empty
\usepackage{emptypage}
%\usepackage{subcaption}
\usepackage{multicol}
\usepackage[dvipsnames]{xcolor}

% Other font I sometimes use.
% \usepackage{cmbright}

% Math stuff
\usepackage{amsmath, amsfonts, mathtools, amsthm, amssymb}
\usepackage{physics}
% Fancy script capitals
\usepackage{mathrsfs}
\usepackage{cancel}
% Bold math
\usepackage{bm}


\usepackage{centernot}
\newcommand{\fbs}[1]{{\centernot{#1}}}

% Some shortcuts
\newcommand\N{\ensuremath{\mathbb{N}}}
\newcommand\R{\ensuremath{\mathbb{R}}}
\newcommand\Z{\ensuremath{\mathbb{Z}}}
\renewcommand\O{\ensuremath{\emptyset}}
\newcommand\Q{\ensuremath{\mathbb{Q}}}
\newcommand\C{\ensuremath{\mathbb{C}}}

% Put x \to \infty below \lim
\let\svlim\lim\def\lim{\svlim\limits}

%Make implies and impliedby shorter
\let\implies\Rightarrow
\let\impliedby\Leftarrow
\let\iff\Leftrightarrow
\let\epsilon\varepsilon
%\let\phi\varphi

\newcommand\up{\uparrow}
\newcommand\down{\downarrow}

% Add \contra symbol to denote contradiction
\usepackage{stmaryrd} % for \lightning
\newcommand\contra{\scalebox{1.5}{$\lightning$}}


% Command for short corrections
% Usage: 1+1=\correct{3}{2}

\definecolor{correct}{HTML}{009900}
\newcommand\correct[2]{\ensuremath{\:}{\color{red}{#1}}\ensuremath{\to }{\color{correct}{#2}}\ensuremath{\:}}
\newcommand\green[1]{{\color{correct}{#1}}}

% horizontal rule
\newcommand\hr{
    \noindent\rule[0.5ex]{\linewidth}{0.5pt}
}

% hide parts
\newcommand\hide[1]{}

% si unitx
\usepackage{siunitx}
\sisetup{locale = UK}

% Environments
\makeatother
% For box around Definition, Theorem, \ldots

% tikz
\usepackage{pgfplots}
\pgfplotsset{compat=1.18}
\usepackage{tikz-3dplot}

\usetikzlibrary{arrows.meta,bending}
\usetikzlibrary{angles,quotes}
\usetikzlibrary{decorations.pathmorphing,patterns}
\usepgfplotslibrary{fillbetween}
\usepackage{tensor}



\usepackage{mdframed}
\mdfsetup{skipabove=20pt,skipbelow=\topsep,leftmargin=\parindent}
\theoremstyle{definition}


\newtheorem*{notation}{Notation}
\newtheorem*{remark}{Remark}
\newtheorem*{note}{Note}
\newtheorem*{problem}{Problem}
\newtheorem*{example}{Example}
\newtheorem*{examples}{Examples}

\newtheorem*{observe}{Observe}
\newtheorem*{property}{Property}
\newtheorem*{intuition}{Intuition}
\newtheorem*{prop}{Proposition}
\newmdtheoremenv[nobreak=true]{law}{Law}

\newtheorem{quiz}{Quiz}
\newtheorem*{claim}{Claim}

\newtheoremstyle{enumstyle}{0}{}{}{}{\bfseries}{:}{ }{\thmname{#1} \thmnumber{#2}\thmnote{#3}}
\theoremstyle{enumstyle}

\newmdtheoremenv[nobreak=true,skipabove=10pt,skipbelow=\topsep]{definition}{Definition}[section]
\newmdtheoremenv[nobreak=true]{theorem}{Theorem}[section]
\newmdtheoremenv[nobreak=true]{corollary}{Corollary}[section]
\newmdtheoremenv[nobreak=true]{lemma}{Lemma}[section]
\newmdtheoremenv[nobreak=true]{principle}{Principle}[section]
\newmdtheoremenv[nobreak=true]{postulate}{Postulate}[section]

\newtheorem{exercise}{Exercise}
\newtheorem{proposition}{Proposition}[section]
%\newtheorem{solution}{Solution}

% Add more box environments here ^^
% \usepackage{mdframed}
% \mdfsetup{skipabove=1em,skipbelow=1em}
% \theoremstyle{definition}
% \newmdtheoremenv[nobreak=true]{definition}{Definition}
% \newmdtheoremenv[nobreak=true]{theorem}{Theorem}

% End example and intermezzo environments with a small diamond (just like proof
% environments end with a small square)
\usepackage{etoolbox}
\AtEndEnvironment{vb}{\null\hfill$\diamond$}%
\AtEndEnvironment{intermezzo}{\null\hfill$\diamond$}%
% \AtEndEnvironment{opmerking}{\null\hfill$\diamond$}%

% Fix some spacing
% http://tex.stackexchange.com/questions/22119/how-can-i-change-the-spacing-before-theorems-with-amsthm
\makeatletter
\def\thm@space@setup{%
  \thm@preskip=\parskip \thm@postskip=0pt
}


% Exercise 
% Usage:
% \oefening{5}
% \suboefening{1}
% \suboefening{2}
% \suboefening{3}
% gives
% Oefening 5
%   Oefening 5.1
%   Oefening 5.2
%   Oefening 5.3



% \lecture starts a new lecture (les in dutch)
%
% Usage:
% \lecture{1}{di 12 feb 2019 16:00}{Inleiding}
%
% This adds a section heading with the number / title of the lecture and a
% margin paragraph with the date.

% I use \dateparts here to hide the year (2019). This way, I can easily parse
% the date of each lecture unambiguously while still having a human-friendly
% short format printed to the pdf.

\usepackage{xifthen}
\def\testdateparts#1{\dateparts#1\relax}
\def\dateparts#1 #2 #3 #4 #5\relax{
    \marginpar{\small\textsf{\mbox{#1 #2 #3 #5}}}
}

\def\@lecture{}%
\newcommand{\lecture}[3]{
    \ifthenelse{\isempty{#3}}{%
        \def\@lecture{Lecture }%
    }{%
        \def\@lecture{Lecture: #3}%
    }%
    \section{\@lecture}

    \marginpar{\small\textsf{\mbox{#2}}}
}



% These are the fancy headers
\usepackage{fancyhdr}
\pagestyle{fancy}







% Todonotes and inline notes in fancy boxes
\usepackage{todonotes}
\usepackage[most]{tcolorbox}
\usepackage{minted}
% Make boxes breakable
\tcbuselibrary{breakable}


\newenvironment{verbetering}{\begin{tcolorbox}[
    arc=0mm,
    colback=white,
    colframe=green!60!black,
    title=Opmerking,
    fonttitle=\sffamily,
    breakable
]}{\end{tcolorbox}}

\newenvironment{noot}[1]{\begin{tcolorbox}[
    arc=0mm,
    colback=white,
    colframe=white!60!black,
    title=#1,
    fonttitle=\sffamily,
    breakable
]}{\end{tcolorbox}}



% Figure support as explained in my blog post.
\usepackage{import}
%\usepackage{xifthen} imported elsewhere
\usepackage{pdfpages}
\usepackage{transparent}
\newcommand{\incfig}[1]{%
    \def\svgwidth{\columnwidth}
    \import{./figures/}{#1.pdf_tex}
}

% Fix some stuff
% %http://tex.stackexchange.com/questions/76273/multiple-pdfs-with-page-group-included-in-a-single-page-warning
%\pdfsuppresswarningpagegroup=1



\newcommand{\axiom}[2]{
    \vb{\left( #1#2 \right)} 
}

\newcommand{\quiztrue}{
    \textcolor{ForestGreen}{True.}
}
\newcommand{\quizfalse}{
    \textcolor{red}{False.}
}
\newcommand{\too}{\longrightarrow}
\newcommand{\mapstoo}{\longmapsto}

\newcommand{\ihat}{\boldsymbol{\hat{\textbf{\i}}}}
\newcommand{\jhat}{\boldsymbol{\hat{\textbf{\j}}}}
\newcommand{\khat}{\boldsymbol{\hat{\textbf{k}}}}
\newcommand{\rhat}{\boldsymbol{\hat{\textbf{r}}}}
\newcommand{\thetahat}{\boldsymbol{\hat{\theta}}}



%\declaretheoremstyle[
%    headfont=\bfseries, 
%    notefont=\scshape, notebraces={(}{)},
%    bodyfont=\itshape,
%    postheadspace=1em,
%    % postheadhook={\textcolor{red}{\rule[.6ex]{\linewidth}{0.4pt}}\\},
%    spacebelow=\topsep,
%    % spaceabove=\topsep,
%    mdframed={
%            % linecolor=black!100, 
%            linewidth=2pt,
%            innerbottommargin=6pt, 
%            skipabove=\topskip, 
%            skipbelow=\topskip,
%            topline=false,
%            bottomline=false, 
%            rightline=false
%            }
%]{proofl}


\renewenvironment{proof}
{ 
    \vspace{6 pt}
    \begin{mdframed}[
        linewidth=1pt,
        skipabove=\topskip,
        skipbelow=0pt,
        innertopmargin=0pt,
        innerbottommargin=6pt,
        bottomline=false,
        topline=false,
        rightline=false]%
    \noindent \textit{\textbf{Proof.}}  
}
{%
    \qed 
\end{mdframed}
    \vspace{6 pt}
}


\mdfdefinestyle{exampledefault}{%
rightline=true,innerleftmargin=10,innerrightmargin=10,
frametitlerule=true,frametitlerulecolor=black,
frametitlebackgroundcolor=lime,
frametitlerulewidth=2pt,
shadow=true,shadowsize=4pt,
}

\newenvironment{solution}
{ 
    \vspace{6 pt}
    \begin{mdframed}[
        linewidth=1pt,
        skipabove=\topskip,
        skipbelow=0pt,
        innertopmargin=0pt,
        innerbottommargin=6pt,
        bottomline=false,
        topline=false,
        rightline=false]%
    \noindent \textbf{Solution.}
}
{%
    %\qed 
\end{mdframed}
    \vspace{6 pt}
}

% PDF hyper-refs see: https://github.com/gillescastel/instant-reference
%\usepackage[hidelinks]{hyperref} imported elsewhere
%\usepackage{xifthen} elsewhere
\usepackage{fontawesome}

\newcommand\urlref[2]{%
    \href{#1}{\raisebox{0.15ex}{\scriptsize \faLink}\:\textup{\textbf{#2}}}%
}

\newcommand\pdfref[3]{%
    \href{ref://open-paper?id=#1&page=#2}{%
    \textup{[\textbf{\ifthenelse{\isempty{#3}}{here}{#3}}]}}%
}




% always capitalize the label with option [capitalize]
\usepackage[capitalize]{cleveref}
\allowdisplaybreaks
% My name
\author{Cian Luke Martin}
