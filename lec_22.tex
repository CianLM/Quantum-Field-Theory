\lecture{22}{29/11/2024}{Covariant Derivatives}

We expect that the propagator is the Green's function for the equation of motion and thus acting the equation of motion on the propagator $\left<T A_\mu \left( x \right) A_\nu \left( y \right)  \right>$ should give a delta function. Therefore
\begin{align}
    \bra{0} T A_{\mu}\left( x \right) A_\nu \left( y \right) \ket{0} &= i \int \frac{\dd{^4p}}{\left( 2\pi\right)^4} e^{-ip \left( x - y \right) } \Pi_{\mu \nu} \\
     &= -i \int \frac{\dd{^4p}}{\left( 2\pi\right)^4} \frac{e^{-ip \left( x - y \right) }}{p^2 + i \epsilon} \left( \eta_{\mu \nu} + \left( \alpha - 1 \right) \frac{p_\mu p_\nu}{p^2} \right)  
.\end{align}

\begin{notes}~
    \begin{itemize}
        \item The minus sign is correct. If $\mu = i$ and $\nu = j$, we will see a positive sign for the physical components which is identical to what we saw for a scalar field.
        \item The presence of $\alpha$ should in principle bother us as it is completely unphysical. The $S$ matrix should not depend on $\alpha$ (and it doesn't as we will see). One can then fix any value of $\alpha$.
        \item Common choices for $\alpha$ include
            \begin{itemize}
                \item $\alpha = 1$, called \textit{Feynman-t'Hooft} gauge,
                \item $\alpha = 0$, called Lorentz (or Landau) gauge, which is a strong enforcement of the Lorentz gauge,
                \item $\alpha \to \infty$, called the unitary gauge, is useful in non-abelian gauge theories.
            \end{itemize}
    \end{itemize}
\end{notes}

\subsection{Interactions: couple light to matter}

The Maxwell equations, $\partial_\mu F^{\mu \nu} = j^{\nu}$ have the consequence that
\begin{align}
    \partial_\nu \partial_\mu F^{\mu \nu} = \partial_\nu j^{\nu} \implies 0 = \partial_\nu j^{\nu}
,\end{align}
namely, that $j^{\nu}$ is a conserved current.

If $j^{\nu}$ is independent of $A_\nu$ itself, then we can write
\begin{align}
    \mathcal{L} = -\frac{1}{4} F_{\mu \nu} F^{\mu \nu} - j^{\mu} A_\mu
.\end{align}

Looking at the action and gauge transforming, we see
\begin{align}
    S = \int \dd{^4x} \mathcal{L} \to \int \dd{^{4}x} \left( -\frac{1}{4} F_{\mu \nu} F^{\mu \nu} - j^{\mu} \left( A_\mu + \partial_\mu \lambda \right)   \right)  = \int \dd{^4x } \mathcal{L} - \int \dd{^4x} \left( \partial_\mu \left( j^{\mu} \lambda \right) - \partial_\mu j^{\mu} \right) \lambda
.\end{align}

Namely, the action is gauge invariant if the current is conserved.

With this, lets couple light to spinors.

Recall the Dirac Lagrangian $\mathcal{L} = \overline{\psi} \left( i \fbs{\partial} - m \right) \psi$ has current
\begin{align}
    j^{\mu} = \overline{\psi}\gamma^{\mu} \psi
,\end{align}
due to the internal symmetry $\psi \to e^{i \alpha} \psi$ (and $\overline{\psi} \to e^{-i\alpha} \psi$).

As this current is conserved, we can couple it to the Maxwell action without breaking gauge invariance. We thus consider Maxwell's theory with spinorial matter and this current coupling them giving
\begin{align}
    \mathcal{L} = -\frac{1}{4} F_{\mu \nu} F^{\mu \nu} + \overline{\psi} \left( i \fbs{\partial} - m  \right) \psi - e j^{\mu} A_\mu
,\end{align}
where we have introduced a coupling constant $e$. Notice that we can equivalently write this as
\begin{align}
    \mathcal{L} &=  -\frac{1}{4} F_{\mu \nu} F^{\mu \nu} + \overline{\psi} \left( i \fbs{\partial} - e A_\mu \gamma^{\mu} - m  \right) \psi \\
    \mathcal{L} &= -\frac{1}{4} F_{\mu \nu} F^{\mu \nu} + \overline{\psi} \left( i \fbs{D} - m \right) \psi 
,\end{align}
where $D_{\mu} = \partial_\mu + i \epsilon A_\mu$ is the \textbf{covariant derivative}.

As one has that the gauge field transforms under a local gauge transformation
\begin{align}
    A_\mu \to A_\mu + \partial_\mu \lambda \left( x \right) 
,\end{align}
it is natural to promote the spinorial $U\left( 1 \right) $ symmetry such that under gauge transformation it undergoes
\begin{align}
    \psi \to e^{-i e \lambda \left( x \right) }\psi \\
    \overline{\psi} \to e^{ie \lambda \left( x \right) }\overline{\psi}
.\end{align}

Then we have
\begin{align}
    D_\mu \psi &\to \partial_\mu \left( e^{-i e \lambda} \psi \right) + i e \left( A_\mu + \partial_\mu \lambda \right) e^{-ie\lambda} \psi \\
    &= e^{-ie \lambda} \left( \partial_\mu -ie \partial_\mu \lambda + ie A_\mu + ie \partial_\mu \lambda \psi \right)  \\
    &= e^{-i e \lambda} D_{\mu}\psi 
,\end{align}
which implies
\begin{align}
    \overline{\psi} \fbs{D} \psi \to \overline{\psi}\fbs{D}\psi
.\end{align}
which makes the action gauge invariant.

Looking at Noether's theorem, we see
\begin{align}
    Q &= e\int \dd{^3x} F^{0i} \partial_i \lambda \\
    &= -e\int \dd{^3x} \partial_i F^{0i} \lambda \\
    \intertext{for $\lambda = 1$,}
    Q &= -e\int \dd{^3x} j^{0} \\
    &= -e \int \dd{^3x} \overline{\psi} \gamma^{0} \psi 
.\end{align}

Gauge symmetries, if they contain a \textit{global symmetry} (i.e. internal symmetry) where the parameter is constant, then we have an associated charge.

\begin{note}
    Constant $\lambda$ is called a \textit{large gauge transformation} as it does not vanish at infinity.
\end{note}



