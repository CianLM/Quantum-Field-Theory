\lecture{4}{18/10/2024}{Canonical Quantization}

\subsection{Internal Symmetry}


\begin{example}[Internal Symmetry]
    Internal symmetries do not act on coordinates, only the fields. Consider a complex scalar field
    \begin{align}
        \psi \left( x \right) = \frac{1}{\sqrt{2} } \left( \phi_{1} \left( x \right) + i \phi_2 \left( x \right)  \right) 
    ,\end{align}
    where $\phi_1 ,\phi_2$ are real scalar fields. A Lagrangian for this field is
    \begin{align}
        \mathcal{L} = \partial_\mu \phi \partial^{\mu} \psi^{*} - V \left( \left| \psi \right|^2 \right) 
    .\end{align}
    The equations of motion for this theory are
    \begin{align}
        \partial_\mu \partial^{\mu} \psi + \pdv{V}{\phi^{*}} = 0, && \partial_\mu \partial^{\mu} \psi^{*} + \pdv{V}{\phi} = 0
    .\end{align}
    The internal symmetry of this system, for constant $\alpha \in \R$,
    \begin{align}
        \phi \left( x \right) \to \phi' \left( x \right) &= e^{i \alpha} \psi \left( x \right) \label{eq:global_phase}  \\
        \phi^{*} \left( x \right) \to \left( \phi^{*} \left( x \right)  \right)' &= e^{-i\alpha} \psi^{*} \left( x \right) 
    ,\end{align}
    under which $\mathcal{L} \to \mathcal{L}' = \mathcal{L}$ and $S \to S' = S$. Here $\alpha$ is the continuous parameter of the transformation, such that
    \begin{align}
        \delta \phi &= \phi' \left( x \right) - \phi \left( x \right)  \\
        &= i \alpha \psi \\
        \delta \psi^{*} &= -i \alpha \psi^{*}
    .\end{align}

    We can construct the current
    \begin{align}
        j^{\mu} = \pdv{\mathcal{L}}{\left( \partial_\mu \psi \right) } \delta \psi + \pdv{\mathcal{L}}{\left( \partial_\mu \psi^{*} \right) } \delta \psi^{*} - \mathcal{J}^{\mu}
    ,\end{align}
    where there is no total derivative term, $\mathcal{J}^{\mu} = 0$. We then have
    \begin{align}
        j^{\mu} &= i \alpha \left( \psi \partial^{\mu} \psi^{*} - \psi^{*} \partial_\mu \psi \right) 
    ,\end{align}
    which implies a conserved charge
    \begin{align}
        Q = \int \dd{^3x} j^{0}
    ,\end{align}
    which is in fact the electric charge as we will see.

    Observe that it is also possible to view the transformation as
    \begin{align}
        \mqty ( \phi_1 \\ \phi_2 ) \to \mqty( \phi_1' \\ \phi_2' ) = \mqty( \cos \alpha & - \sin \alpha \\ \sin \alpha & \cos \alpha ) \mqty( \phi_1 \\ \phi_2 )
    ,\end{align}
    which is identical to the previous transformation in \cref{eq:global_phase}.

\end{example}

\subsection{Quantum Fields}

We will first study the simplest possible theory: a free theory. We will take a Hamiltonian approach and build on the rules of quantum mechanics. Recall the familiar commutation relations of
\begin{align}
    \left[ x^{i}, p^{j} \right] = i \delta^{ij}
.\end{align}

In QFT, we no longer speak of position and momentum variables, but rather a quantum field $\phi_a \left( x \right) $ and its conjugate momenta $\Pi^{a}\left( x \right) = \pdv{\mathcal{L}}{\dot{\phi}_a}$ which satisfy
\begin{align}
    \left[ \phi_a \left( \vb{x},t \right) , \Pi^{b}\left( \vb{y},t \right)  \right] = i \delta^{3} \left( \vb{x} - \vb{y} \right) \delta^{b}_a
,\end{align}
called \textit{equal time} commutation relations. One must make a choice of some kind when transferring from a classical theory to a quantum theory, and this turns out to be one such correct choice.

% hamiltonian
% fock space
% causality
% propagators

\subsection{Canonical Quantization}

\begin{note}
    In the notes, Tong performs canonical quantization in the Schrödinger picture at $t = 0$. Here we will use the Heisenberg picture.
\end{note}

Our theory of interest is
\begin{align}
    \mathcal{L} = \frac{1}{2} \partial_\mu \phi \partial^{\mu} \phi - \frac{1}{2}m^2 \phi^2
.\end{align}

Its equation of motion is the Klein-Gordon equation, $\partial_\mu \partial^{\mu} \phi + m^2 \phi = 0$.

We know solutions to this equation take the form
\begin{align}
    \phi \sim  \exp \left( i \vb{k} \cdot \vb{x} + i \omega t \right) 
,\end{align}
where $- \omega^2 + \vb{k} \cdot \vb{k} + m^2 = 0$ which gives us a dispersion relation,
\begin{align}
    \omega \left( k \right)  = \pm \sqrt{\vb{k} \cdot \vb{k} + m^2} 
.\end{align}

We adopt the notation $\omega \equiv \sqrt{\vb{k} \cdot \vb{k} + m^2}$. Therefore, taking a linear superposition of fields, one has
\begin{align}
    \phi \left( \vb{x},t \right) = \int \frac{\dd{^3k}}{\left( 2\pi \right)^3} \left( a \left( k \right) e^{i \vb{k} \cdot \vb{x} - i \omega t}  + b\left( k \right) e^{i \vb{k} \cdot \vb{x} + i \omega t}\right) 
.\end{align}

\begin{note}
    $\phi$ is real, which imposes restrictions on $a\left( k \right) $ and $b\left( k \right) $. Namely, as $\phi^{*} = \phi$, we have
    \begin{align}
        a^{*}\left( -k \right) = b\left( k \right)  && b^{*}\left( -k \right) = a \left( k \right) 
    ,\end{align}
    thus we can write
    \begin{align}
        \phi \left( x \right) = \int \frac{\dd{^3k}}{\left( 2\pi \right)^3} \left( a \left( k \right) e^{i \vb{k} \cdot \vb{x} - i \omega t} + a^{*} \left( k \right) e^{-i \vb{k} \cdot \vb{x} + i \omega t}  \right) 
    .\end{align}
    In a more relativistic notation, one has
    \begin{align}
        \phi \left( x \right) = \int \frac{\dd{^3k}}{\left( 2\pi \right)^3} \left( a \left( k \right) e^{-i k^{\mu} x_{\mu}} + a^{*}\left( k \right) e^{i k^{\mu} x_\mu} \right) 
    ,\end{align}
    where $k_{\mu} = \left(\omega, \vb{k}\right) $ and $x_{\mu} = \left( t, \vb{x} \right)  $ give us $k^{\mu} x_\mu = \omega t - \vb{k} \cdot \vb{x}$ and $k^2 = \omega^2 - \vb{k} \cdot \vb{k} = m^2$.
\end{note}

\begin{note}
    We will choose to normalize $a\left( k \right) $ and $a^{*}\left( k \right) $ such that
    \begin{align}
        \phi \left( x \right) = \int \frac{\dd{^3k}}{\left( 2\pi \right)^3} \frac{1}{\sqrt{2 \omega} } \left( a \left( k \right) e^{-i k_\mu x^{\mu}} + a^{*}\left( k \right) e^{i k_\mu x^{\mu}} \right) 
    .\end{align}
\end{note}

Lastly, notice that
\begin{align}
    \Pi \left( x \right) = \dot{\phi} = \int \frac{\dd{^3k}}{\left( 2\pi \right)^3} -i \sqrt{\frac{\omega}{2}} \left( a \left( k \right) e^{-i k_\mu x^{\mu}} - a^{*}\left( k \right) e^{i k_\mu x^{\mu}} \right)  
.\end{align}

Next, we \textit{\textbf{quantize}}, namely, we declare that
\begin{align}
    \left[ \phi \left( \vec{x},t \right) , \phi \left( \vec{x}',t \right)  \right] &= 0 \\
    \left[ \Pi \left( \vec{x},t \right) , \Pi \left( \vec{x}',t \right)  \right] &= 0 \\
    \left[ \phi \left( \vec{x},t \right) , \Pi \left( \vec{x}', t \right)  \right] &= i \delta^{3} \left( \vec{x} - \vec{x}' \right) 
.\end{align}

\begin{claim}
    These commutation relations promote $a$ to an \textbf{operator} such that $a \left( k \right) $ becomes $\hat{a}_{k}$ and $a^{*} \left( k \right)$ becomes $\hat{a}_{k}^{\dag}$. The above commutation relations imply
\begin{align}
    \left[ \hat{a}_k, \hat{a}_{k'} \right] &= 0 \\ 
    \left[ \hat{a}_k^{\dag}, \hat{a}^{\dag}_{k'} \right] &= 0 \\
    \left[ \hat{a}_k, \hat{a}_{k'}^{\dag} \right] &= \left( 2\pi \right)^3 \delta^{3} \left( k - k' \right) 
.\end{align}
\end{claim}

\begin{proof}
    \begin{enumerate}[label=\arabic*)]
        \item (\textit{Claim implies declaration}) Taking
            \begin{align}
                \left[ \phi \left( \vec{x},t \right) , \Pi \left( \vec{y},t \right)  \right] &= \int \frac{\dd{^3p} \dd{^3q}}{\left( 2\pi \right)^{6}} \frac{1}{2i} \sqrt{\frac{\omega_q}{\omega_p}}\left(  \left[ a_{\vb{p}} e^{i \vec{p} \cdot \vec{x} + i \omega t} + a^{\dag}_{\vb{p}} e^{-i \vec{p} \cdot \vec{x} + i \omega t}, a_{\vb{q}} e^{i \vec{q} \cdot \vec{y} - i \omega t} - a^{\dag}_{\vb{q}} e^{-i \vec{q} \cdot \vec{y} + i \omega t}  \right]  \right) \\
&= \int \frac{\dd{^3p} \dd{^3q}}{\left( 2\pi \right)^{6}} \frac{1}{2i} \sqrt{\frac{\omega_q}{\omega_p}} \left( - \underbrace{\left[ a_{\vb{p}}, a^{\dag}_{\vb{q}} \right]}_{\left( 2\pi \right)^3 \delta^{3}\left( \vec{p} - \vec{q} \right) } e^{i \vec{p} \cdot \vec{x}} e^{-i \vec{q} \cdot \vec{y}} + \left[ a^{\dag}_{\vec{p}}, a_{\vb{q}} \right] e^{-i \vec{p} \cdot \vec{x}} e^{i \vec{q} \cdot \vec{y}}  \right) \\
&= i\int \frac{\dd{^3p}}{\left( 2\pi \right)^{3}} e^{i \vec{p} \cdot \left( \vec{x} - \vec{y} \right) } = i \delta^{3}\left( \vec{x} - \vec{y} \right)
            ,\end{align}
            as desired.
    \end{enumerate}
\end{proof}


