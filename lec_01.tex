\lecture{1}{11/10/2024}{Introduction}

Historically, the goal of quantum field theory was to combine quantum mechanics with special relativity. One of the most notable outputs of this study hailed as a success is that the number of particles is not conserved. It is a robust and systematic theory governed by few principles. It concerns itself with locality, symmetries and renormalization which are exceptionally constraining and \textit{almost} uniquely determine what we can study. 

In this course we use $c = \hbar = 1$. In these natural units, $E = mc^2$ gives us masses in the units of energy.

For metrics we use $\eta^{\mu \nu} = \text{diag}\{1,-1,-1,-1\} $, $x^{\mu} = \left( t,x,y,z \right) $, $F \left( t,\vec{x} \right) \equiv F\left( x^{\mu} \right) \equiv F \left( \vb{x} \right) $

\subsection{Classical Field Theory}


In classical mechanics, a natural object is the action,
\begin{align}
    S\left( t_1,t_2 \right) = \int_{t_1}^{t_2} \dd{t} \left( \underbrace{m \sum_{i=1}^{3} \left( \dv{x^{i}}{t} \right)^2}_{\text{kinetic term}} - \underbrace{V\left( x \right)}_{\text{potential}}    \right) 
.\end{align}

This is incredibly useful for us for three main reasons:
\begin{itemize}
    \item the equations of motion are given for free by extremising $S$,
    \item Boundary conditions are supplied externally, and
    \item $S$ is built on \textit{symmetry} (it is invariant of symmetries of your system).
\end{itemize} 

As we move towards field theory, we no longer want to speak of a single position of a particle $x\left( t \right) $.

The fundamental object in field theory is a field $\phi_a \left( t, \vec{x} \right) : \R^{3,1} \to \R$ or $\C$ or $\R^{n}$. Here $a$ labels the type of field we are discussing.


The first consequence is that we are dealing with an infinite number of degrees of freedom as every point in time and space contains some distinct information about the system.

\begin{example}
    In electromagnetism, as we will discuss in depth later, one has the gauge field $A^{\mu} \left( t, \vec{x} \right) = \left( \phi \left( x \right) , \vec{A}\left( x \right)  \right) $ which the electric and magnetic fields can be defined in terms of
    \begin{align}
        \vec{E} &= - \grad \phi - \pdv{\vec{A}}{t} \\
        \vec{B} &= \grad \cdot \vec{A}
    ,\end{align}
    which have equations of motion
    \begin{align}
        \grad \cdot \vb{E} &= \rho  \\
        \grad \times  \vec{B} = \vec{J} + \pdv{\vec{E}}{t}
    ,\end{align}
    and two identities
\begin{align}
    \grad \cdot \vec{B} &= 0 \\
    \dv{\vec{B}}{t} = - \grad \times \vec{E}
.\end{align}

This is a (hopefully) familiar classical field that we will quantise in due time.
\end{example}


\subsection{Lagrangians}

The Lagrangian in classical mechanics can be written $L = T - V$ and is contained within the action in the form
\begin{align}
    S = \int \dd{t} L
.\end{align}

We will in QFT concern ourselves with the \textit{Lagrangian density} given by
\begin{align}
    L = \int \dd{^3x} \mathcal{L} \left( \phi_a , \partial_\mu \phi_a \right) 
,\end{align}
however \textit{everybody} just refers to $\mathcal{L}$ as a Lagrangian as we will here.

The equations of motion are determined by extremizing with respect to the fields. 

\begin{note}
    Note that we assume that the Lagrangian $\mathcal{L}\left[ \phi_a, \partial_\mu \phi_a \right] $ is not a function of $\partial^2 \phi_a$ or higher derivatives. This is for complicated reasons related to ghosts that are beyond the scope of this course.
\end{note}

Extremising the action with respect to the field, we want to find the conditions for which $\Delta S = 0$, i.e. the action is at a minima/saddle point. We see that
\begin{align}
    \delta S &= \int \dd{^{4}x} \left[ \pdv{\mathcal{L}}{\phi_a} \delta \phi_a + \pdv{\mathcal{L}}{\left( \partial_\mu \phi_a \right) } \delta \left( \partial_\mu \phi_a \right)  \right] \\
    &= \int \dd{^{4}x} \left[ \pdv{\mathcal{L}}{\phi_a} \delta \phi_a - \partial_\mu \left( \pdv{\mathcal{L}}{\left( \partial_\mu \phi_a \right) }\right)\delta \phi_a   + \underbrace{\partial_\mu \left( \pdv{\mathcal{L}}{\left( \partial_\mu \phi_a \right) } \delta \phi_a \right)}_{\text{total derivative}}   \right]
,\end{align}
and by assuming that our fields decay at infinity, the total derivative term vanishes yielding
\begin{align}
   \delta S &= \int \dd{^{4}x} \left[ \pdv{\mathcal{L}}{\phi_a}  - \partial_\mu \left( \pdv{\mathcal{L}}{\left( \partial_\mu \phi_a \right) } \right) \right]\delta \phi_a 
,\end{align}
for which vanishing requires
\begin{align}
    \partial_\mu \left( \pdv{\mathcal{L}}{\left( \partial_\mu \phi_a \right) } \right) - \pdv{\mathcal{L}}{\phi_a} = 0
.\end{align}

\begin{example}
    A free massive scalar field is described by the Lagrangian
    \begin{align}
        \mathcal{L} &= \frac{1}{2} \eta^{\mu \nu} \partial_\mu \phi \partial_\nu - \frac{1}{2} m^2 \phi^2 \\
        &= \frac{1}{2} \dot{\phi}^2 - \frac{1}{2} \left( \grad \phi \right)^2 - \frac{1}{2}m^2 \phi^2
    .\end{align}

    In traditional classical mechanics, one would have identified $T = \frac{1}{2} \dot{\phi}^2$ and $V = \frac{1}{2} \left( \grad \phi \right)^2 + \frac{1}{2} m^2 \phi^2$. 
    In QFT, the `kinetic terms' sometimes refers to any bilinear combination of fields. For example, $\eta^{\mu \nu} \partial_\mu \phi \partial_\nu \phi$ is always kinetic and $m^2 \phi^2$ is often a (bosonic) mass term.

    The equation of motion for the free massive scalar field Lagrangian is
    \begin{align}
        \partial_\mu \partial^{\mu} \phi + m^2 \phi = 0
    .\end{align}
    This is the \textbf{Klein Gordon equation}. It is also sometimes written with $\partial_\mu \partial^{\mu} = \Box$.
\end{example}



