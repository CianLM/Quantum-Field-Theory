\lecture{10}{01/11/2024}{Interactions}

\subsection{Schwinger-Dyson Formula}

It remains for us to figure out a way to evaluate
\begin{align}
    \bra{\Omega} T \phi \left( x_1 \right) \cdots \phi \left( x_{n} \right) \ket{\Omega}
.\end{align}

Our strategy is to present a Lagrangian approach to this. We first assume that at any given time, the Hilbert space of the interacting theory is the Hilbert space of the free theory. This implies that
\begin{align}
    \left[ \phi \left( \vb{x},t \right) , \phi \left( \vb{x}', t \right)  \right] &= 0
,\end{align}
and
\begin{align}
    \left[ \phi \left( \vb{x},t \right) , \partial_t \phi \left( \vb{x}', t \right)  \right] &= i \delta^3 \left( \vb{x} -\vb{x}' \right) 
.\end{align}

We also need to assume that our fields still comply with the Euler-Lagrange equations.

For the free theory, this was the Klein Gordon equation,
\begin{align}
    \left( \Box + m^2 \right) \phi = 0
,\end{align}
and for the interacting theory, it takes the form
\begin{align}
    \left( \Box + m^2 \right) \phi - \pdv{\mathcal{L}_\text{int}'}{\phi} = 0
,\end{align}
as we assume $\mathcal{L}_{\text{int}}$ is a function of $\phi$ but not $\partial_\mu \phi$.

\begin{note}
    In a Hamiltonian derivation you would assume
    \begin{align}
        \partial_t \phi = i \left[ H, \phi \right] 
    .\end{align}

    Also note that we will use the notation
    \begin{align}
        \bra{\Omega} T \phi \left( x_1 \right) \cdots \phi \left( x_{n} \right) \ket{\Omega} \equiv \left<\phi_1 \cdots \phi_n \right>
    ,\end{align}
    where we assume the expectation value is taken in a time ordered fashion with respect to the interacting vacuum. We also use $\phi_1 \equiv \phi \left( x_1 \right) $.
\end{note}

\begin{claim}
    \begin{align}
        \left( \Box_x + m^2 \right) \left<\phi_x \phi_y \right> = \left< \left( \Box_x + m^2 \right) \phi_x \phi_y \right> - i \delta^{4} \left(  x- y \right) 
    .\end{align}
\end{claim}

\begin{proof}
    As a warm up, lets study the free theory for which
    \begin{align}
        \left( \Box_x + m^2 \right) \underbrace{\left<\phi_x^{0} \phi_y^{0} \right>}_{\Delta_F \left( x- y \right) } = 0 - i \delta^{4}\left( x-y \right) 
    ,\end{align}
    which we have already established in the free theory. For the interacting theory,
    \begin{align}
        \partial_{x^0} \left< \phi_x \phi_y \right> &= \partial_{x^0} \left( \bra{\Omega} \phi_x \phi_y \ket{\Omega} \Theta \left( x^{0} - y^{0} \right) + \bra{\Omega} \phi_y \phi_x \ket{\Omega} \Theta \left( y^{0} - x^{0} \right)  \right)  \\
        &= \left<\partial_{x^{0}} \phi_x \phi_y \right> + \bra{\Omega} \phi_x \phi_y \ket{\Omega} \partial_x \Theta \left( x^{0} - y^{0} \right)  + \bra{\Omega} \phi_y \phi_x \ket{\Omega} \partial_{x^{0}} \Theta \left( y^{0} - x^{0} \right)  \\
        &= \left< \partial_{x^{0}} \phi_x \phi_y \right> + \delta \left( x^{0} - y^{0} \right) \bra{\Omega} \underbrace{\left[ \phi_x, \phi_y \right]}_{0} \ket{\Omega}
    ,\end{align}
    where the commutator vanishes as we have equal time.
    Then notice,
    \begin{align}
        \partial_{x^{0}}^2 \bra{\phi_x \phi_y} &= \left<\partial_{x^{0}}^2 \phi_x \phi_y \right> + \delta \left( x^{0} - y^{0} \right) \bra{\Omega} \underbrace{\left[ \partial_{x^{0}} \phi_x , \phi_y \right]}_{-i \delta^{3}\left( \vb{x} - \vb{y} \right) } \ket{\Omega} \\
        &= \left<\partial_{x^{0}}^2 \phi_x \phi_y \right> - i \delta^{4} \left( x -y \right) 
    .\end{align}

    As the spatial derivatives and mass terms do nothing, we arrive at the claim.
\end{proof}

This is a first example of an expression we will call the \textit{Schwinger-Dyson equation}. It can be generalized such that
\begin{align} \label{eq:schwinger_dyson}
    \left( \Box_x + m^2 \right) \left<\phi_x \phi_1 \cdots \phi_n \right> &= \left< \pdv{\mathcal{L}_{\text{int}}\left( \phi \left( x \right) \right)}{\phi}   \phi_1 \cdots \phi_n \right> - i \sum_{j=1}^{n}  \delta^{4} \left( x - x_j \right) \left<\phi_1 \cdots \phi_{j-1} \phi_{j+1} \cdots \phi_n \right>
.\end{align}

\begin{example}
    Observe that for the four point function in the free theory, based on Wick's theorem, we expect
    \begin{align}
        \left<\phi_1^{0} \phi_2^{0} \phi_3^{0} \phi_4^{0} \right> &\underset{\text{Wick's}}{=} \Delta_{F} \left( x_1 - x_2 \right) \Delta_{F}\left( x_3 - x_4 \right) + \Delta_F \left( x_1 - x_3 \right) \Delta \left( x_2 - x_4 \right) + \Delta_F \left( x_1 - x_4 \right) \Delta \left( x_2 - x_3 \right) \nonumber \\
        &~~\,\equiv \Delta_{12} \Delta_{34} + \Delta_{13} \Delta_{24} + \Delta_{14} \Delta_{23}
    .\end{align}
    On the contrary, if we derive via Schwinger-Dyson (and dropping the superscripts but still working with the free theory), we see
    \begin{align}
        \left< \phi_1 \phi_2 \phi_3 \phi_4  \right>&= \int \dd{^{3}x} \delta \left( x - y \right) \left< \phi_x \phi_2 \phi_3 \phi_4\right> 
    .\end{align}
    As $\delta^{4}\left( x - x_1 \right) = \left( \Box_x + m^2 \right)  \Delta_{x1}$,
    \begin{align}
        \left< \phi_1 \phi_2 \phi_3 \phi_4  \right> &= i\int \dd{^{3}x} \left( \left( \Box_x + m^2 \right)  \Delta_{x 1} \right)  \left< \phi_x \phi_2 \phi_3 \phi_4\right>  \\
        &= i\int \dd{^{3}x} \Delta_{x 1}\left( \left( \Box_x + m^2 \right)     \left< \phi_x \phi_2 \phi_3 \phi_4\right> \right) \\
        &= i\int \dd{^{3}x} \Delta_{x 1} \left( -i \delta \left( x - x_2 \right) \left<\phi_3 \phi_4 \right> - i \delta \left( x - x_3 \right) \left<\phi_2 \phi_4 \right> - i \delta \left( x - x_4 \right) \left<\phi_2 \phi_3 \right> \right)     \\
        &= \Delta_{12} \Delta_{34} + \Delta_{13} \Delta_{24} + \Delta_{14} \Delta_{23} 
    ,\end{align}
    which agrees with the expression we obtained using Wick's theorem.
\end{example}

\begin{example}
    Consider a cubic interaction, $\mathcal{L}_\text{int} = \frac{g}{3!} \phi^3$. We will compute the one point function,
    \begin{align}
        \left<\phi_x \right> &= \int \dd{^{4}y} \delta \left( x - y \right) \left<\phi_y \right> \\
        &= i\int \dd{^{4}y} \left( \Box_{y} + m^2 \right)  \Delta_{xy} \left<\phi_y \right> \\
        \intertext{Integrating by parts, we see}
        \left<\phi_x \right>&= i\int \dd{^{4}y} \Delta_{xy}\left( \Box_{y} + m^2 \right)   \left<\phi_y \right> \\
        \intertext{And using the Schwinger-Dyson equation,}
        \left<\phi_x \right>&= i\int \dd{^{4}y} \Delta_{xy} \frac{g}{2} \left<\phi_y^2 \right>
        \intertext{Expanding perturbatively in $g$, we see}
        \left<\phi_x \right>&= \frac{ig}{2}\int \dd{^{4}y} \Delta_{xy} \left<\left( \phi_y^{0} \right)^2 \right> + \mathcal{O}\left( g^3 \right)  \\
        \left<\phi_x \right>&= \frac{ig}{2}\int \dd{^{4}y} \Delta_{xy} \Delta_{yy} + \mathcal{O}\left( g^3 \right) \\
        &= \frac{ig}{2} \left( \text{tadpole diagram} \right)  + \mathcal{O}\left( g^3 \right) 
    .\end{align}
\end{example}
