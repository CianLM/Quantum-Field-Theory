\lecture{11}{04/11/2024}{Feynman diagrams}

\begin{example}
    Calculating the three point function in this $\phi^3$ theory, we see
    \begin{align}
        \left<\phi_1 \phi_2 \phi_3 \right> &= \int \dd{^{4}x} \delta \left( x - x_1 \right) \left< \phi_x \phi_2 \phi_3 \right> \\
        &= i \int \dd{^{4}x} \left( \Box_x + m^2 \right) \left<\phi_x \phi_2 \phi_3 \right> \\
        &= \frac{ig}{2} \int \dd{^{4}x} \Delta_{x 1} \left< \phi_x \phi_x \phi_2 \phi_3 \right> + \int \dd{^{4}x} \Delta_{x 1} \left( \delta \left( x - x_2 \right) \left<x_3 \right> + \delta \left( x - x_3 \right) \left<\phi_2 \right> \right)
      .\end{align}
      Using the expression for the four point function in the free theory, and the expression for the one point function, we see
      \begin{align}
          \left<\phi_1 \phi_2 \phi_3 \right> &= \frac{ig}{2} \int \dd{^{4}x} \Delta_{x 1} \left( \Delta_{x x} \Delta_{23} + 2 \Delta_{x 3} \Delta_{x 2} \right) + \frac{ig}{2} \int \dd{^{4}x} \Delta_{x 1} \delta \left( x - x_2 \right) \int \dd{^{4}y} \Delta_{3y} \Delta_{y y} \nonumber  \\
          &\quad + \frac{ig}{2} \int \dd{^{4}x} \Delta_{x 1} \delta \left( x - x_3 \right) \int \dd{^{4}y} \Delta_{2 y} \Delta_{y y} \\
          &= ig \int \dd{^{4}x} \Delta_{x 1 } \Delta_{x 3} \Delta_{x 2} + \frac{ig}{2} \int \dd{^{4}x} \Delta_{x x} \left(  \Delta_{x 1} \Delta_{2 3} + \Delta_{12} \Delta_{3 x} + \Delta_{1 3} \Delta_{2 x} \right) + \mathcal{O}\left( g^2 \right) \\
          &= ig \text{diagram} + \frac{ig}{2} \left( \text{diagram} + \cdots \right)
      .\end{align}
\end{example}

\begin{example}
    Returning to a two point function, we see
    \begin{align}
        \left<\phi_1 \phi_2 \right> &= i \int \dd{^{4}x} \Delta_{1 x} \left( \frac{g}{2} \left< \phi_x^2 \phi_2 \right> - i \delta \left( x - x_2 \right)  \right)  \\
        &= \Delta_{12} + \frac{ig}{2} \int \dd{^{4}x} \dd{^{4}y} \delta \left( y - x_2 \right)  \Delta_{1x} \left<\phi_x^2 \phi_y \right> \\
        &= \Delta_{12} + \frac{ig}{2} \int \dd{^{4}x} \dd{^{4}y} i \Delta_{1 x} \Delta_{2 y} \left( \frac{g}{2} \left<\phi_x^2 \phi_y^2 \right> - 2i \delta \left(  x - y \right) \left<\phi_x \right>\right)  \\
        &= \Delta_{12} + \frac{\left( ig \right)^2}{4} \int \dd{^{4}x} \dd{^{4}y} \left( \Delta_{1x} \Delta_{2y} \Delta_{x x} \Delta_{y y} + 2 \Delta_{1 x} \Delta_{2 y} \Delta_{x y} \Delta_{x y} + 2\Delta_{1 x} \Delta_{2 x} \Delta_{x y} \Delta_{y y} \right) + \mathcal{O}\left( g^3 \right)
    .\end{align}
    
\end{example}

\subsection{Feynman Diagrams}

The Feynman rules for calculating $\left<\phi_1 \cdots \phi_n \right>$ in Scalar Yukawa Theory are as follows:
\begin{enumerate}[label=\arabic*)]
    \item Start with $x_{i}$ external points. Draw a line from each point%: figure
    \item A line can either:
        \begin{itemize}
            \item contract an existing line giving $\Delta_F \left( x_i - x_j \right)$,
            \item split giving a new vertex, where the coefficient will be $i \lambda_n$ for $\mathcal{L}_\text{int}= \frac{\lambda_{n}}{n!} \phi^{n}$. The number of lines depends on $\mathcal{L}_\text{int}'$.
        \end{itemize}
    \item At any given order in $i \lambda_n$, the result is the sum of all diagrams with all lines contracted and integrated over vertices.
    \item Warning: symmetry factors.
\end{enumerate}
