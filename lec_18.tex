\lecture{18}{20/11/2024}{Dirac Interactions}

When normal ordering fermionic operators, note that there is a sign as
\begin{align}
    \nord{b_q c_q b_c^{\dag}} &= \left( -1 \right)^2 b_c^{\dag} b_q c_q \\
    &= \left( -1 \right) b_c^{\dag} c_q b_q 
.\end{align}

We now define the Fock space. We define a vacuum state $\ket{0}$ such that
\begin{align}
    b_p^{s} \ket{0} = c_p^{s} \ket{0} = 0
,\end{align}
and observe $H \ket{0}= 0$ is a ground state.

We have one particle states
\begin{align}
    \ket{b^{s}} = \sqrt{2\omega_p} \left( b_p^{s} \right)^{\dag} \ket{0}  && H \ket{b^{s}} = \omega_p \ket{b^{s}}, \\
    \ket{c^{s}} = \sqrt{2\omega_p}  \left( c^{s}_p \right)^{\dag} \ket{0} && H \ket{c^{s}} = \omega_p \ket{c^{s}}
.\end{align}

Recall that we have $U\left( 1 \right) $ charge given by
\begin{align}
    Q = \sum_{s}^{} \int \frac{\dd{^3p}}{\left( 2\pi\right)^3} \left( \left( b^{s}_p \right)^{\dag} b_p^{s} - \left( c^{s}_p \right)^{\dag} c_p^{s} \right) 
,\end{align}
where observe
\begin{align}
    Q \ket{b^{s}} = \ket{b^{s}} && Q \ket{c^{s}} = - \ket{c^{s}}
.\end{align}

One can also write down an operator for the angular momentum $J_z$ at $\vec{p} = 0$ and see
\begin{align}
    J_{z} \ket{b^{s}} = \pm \frac{1}{2} \ket{b^{s}} && J_z \ket{c^{s}} = \pm \frac{1}{2} \ket{c^{s}}
,\end{align}
where $s = 1 $ gives $+$ and $s = 2$ gives $-$.

\begin{proof}
    Do this exercise. You need $ J_z = \frac{1}{2}\epsilon_{ij 3} \left( j^{0} \right)^{ij}$ where $j^{0} = i S^{ij} \overline{\psi} \gamma^{0} \psi$, $u_{-p}^{s^{\dag}} v_{p}^{s'} = v_{p}^{s^{\dag}} u_{-p}^{s'} = 0$ and $u^{r^{\dag}}_{p} u^{s}_p = v^{r^{\dag}}_p v_p^{s} =  2 \omega_p \delta^{rs}$ as well as the chiral representation of $\gamma^{\mu}$.
\end{proof}

We identically have multiparticle states by acting with multiple $b^{\dag}$'s and $c^{\dag}$'s. For example,
\begin{align}
    \ket{p_1, s_1 ; p_2 , s_2} = \left( b_{p_1}^{s_1} \right)^{\dag} \left( b_{p_2}^{s_2} \right)^{\dag} \ket{0} = -  \left( b_{p_2}^{s_2} \right)^{\dag} \left( b_{p_1}^{s_1} \right)^{\dag} \ket{0} = - \ket{p_2, s_2 ; p_1, s_1}
.\end{align}

We now move to derive the propagator. We inspect first $\bra{0} \psi_a \left( x \right) \overline{\psi}_b \left( y \right) \ket{0}$ with the aim of getting $\bra{0} T \psi \left( x \right) \overline{\psi}\left( y \right) \ket{0}$, the Feynman propagator for the Dirac spinor. This should be Lorentz invariant.

Recall that
\begin{align}
    \psi \left( x \right) &= \sum_{s}^{} \int \frac{\dd{^3p}}{\left( 2\pi\right)^3} \frac{1}{\sqrt{2\omega_p} } \left( b_p^{s} u^{s}\left( p \right) e^{-ipx} + \left( c_p^{s} \right)^{\dag} v^{s}\left( p \right) e^{i px} \right) 
,\end{align}
and similarly for $\overline{\psi}$. One sees then that
\begin{align}
    \bra{0} \psi_a \left( x \right) \overline{\psi}_b \left( y \right) \ket{0} &= \sum_{s,s'}^{} \int \frac{\dd{^3p}}{\left( 2\pi\right)^3} \int \frac{\dd{^3p'}}{\left( 2\pi\right)^3} e^{-ipx} e^{i p'y} u_a^{s}\left( p \right) \overline{u}_b^{s'}\left( p' \right) \bra{0} b_p^{s} \left( b_{p'}^{s'} \right)^{\dag} \ket{0} \\
    &= \int \frac{\dd{^3p}}{\left( 2\pi\right)^3} \frac{1}{2\omega_p} \left( \fbs{p} + m \right)_{ab} e^{-i p \left( x -y \right) } \\
    &=  \left( i\fbs{\partial}_x + m \right) \int \frac{\dd{^3p}}{\left( 2\pi\right)^3} \frac{1}{2\omega_p} e^{-i p \left( x -y \right) } \\
    &= \left( i \fbs{\partial}_x + m \right) D \left( x - y \right) 
,\end{align}
where $D \left( x -y \right) $ is the scalar propagator.

Looking at the reverse we see
\begin{align}
    \bra{0} \overline{\psi}_b \left( y \right) \psi_a \left( x \right) \ket{0} &= \int \frac{\dd{^3 p}}{\left( 2\pi\right)^3} \frac{1}{2\omega_p} \left( \fbs{p} - m \right)  e^{-i p \left( y - x \right) } \\
    &= - \left( i  \fbs{\partial}_x + m \right) D \left( y - x \right) 
.\end{align}

This minus sign is an indication that we should define time ordering such that
\begin{align}
    T \psi_a \left( x \right) \overline{\psi}_b \left( y \right) = \begin{cases}
        \psi_a \left( x \right) \overline{\psi}_b \left( y \right), & x^{0} > y^{0}, \\
        -\overline{\psi}_b \left( y \right) \psi_a \left( x \right) , & y^{0} > x^{0}.
    \end{cases}
\end{align}

With this, we see
\begin{align}
    \bra{0} T \psi \left( x \right) \overline{\psi}\left( y \right) \ket{0} = \left( i\fbs{\partial}_x + m \right) \Delta_F \left( x - y \right) \equiv S_F \left( x - y \right) 
.\end{align}

It follows that
\begin{align}
    \left( i  \fbs{\partial}_x - m\right) S_F \left( x - y \right) = i \delta^{4}\left( x - y \right) 
,\end{align}
as $\left( \Box_x + m^2 \right) \Delta_F \left( x-  y \right) = -i \delta^{4} \left( x - y \right) $.

Also,
\begin{align}
    S_F \left( x - y \right)\left( i  \overset{\leftharpoonup}{\fbs{\partial}}_x + m\right)  = -i \delta^{4}\left( x - y \right) 
.\end{align}

\begin{theorem}[ (Wick's Theorem for spinors)]
    Observe
    \begin{align}
        T \left( \psi \left( x \right) \overline{\psi}\left( y \right)  \right) = \nord{\psi \left( x \right) \overline{\psi}\left( y \right) } + \wick{ \c {\overline{\psi}} \c \psi}
    ,\end{align}
    where $\wick{ \c {\overline{\psi}} \c \psi} = S_F \left( x - y \right) $
\end{theorem}

\begin{note}
    If $x^{0}_3 > x^{0}_1 > x^{0}_4 > x_2^{0}$,
    \begin{align}
        T \left( \psi_1 \psi_2 \psi_3 \psi_4 \right) = \left( -1 \right)^{3} \psi_3 \psi_1 \psi_4\psi_2
    .\end{align}
\end{note}

\subsection{Interactions: Yukawa theory}

Given the Lagrangian
\begin{align}
    \mathcal{L}_0 = \frac{1}{2} \partial_\mu \phi \partial^{\mu} \phi - \frac{1}{2} \mu^2 \phi^2 + \overline{\psi} \left( i \gamma^{\mu} \partial_\mu - m \right) \psi
,\end{align}
and the interaction term
\begin{align}
    \mathcal{L}_\text{int} = -\lambda \phi \overline{\psi}\psi
.\end{align}

Consider nucleon-anti nucleon scattering with
\begin{align}
    \ket{i, - \infty} = \sqrt{2\omega_p}  \sqrt{2\omega_q} \left( b_p^{s} \right)^{\dag} \left( -\infty \right) \left( c_q^{r} \right)^{\dag} \left( -\infty \right) \ket{\Omega}  
,\end{align}
and
\begin{align}
    \ket{f, \infty} = \sqrt{2\omega_{p'}} \sqrt{2\omega_{q'}} \left( b_{p'}^{s'} \right)^{\dag} \left( \infty \right)  \left( b_{q'}^{r'} \right)^{\dag} \left( \infty \right) \ket{\Omega}    
.\end{align}

Then we have
\begin{align}
    \bra{f} S \ket{i} = \sqrt{2\omega_p}  \sqrt{2\omega_{p'}} \sqrt{2\omega_{q}}  \sqrt{2 \omega_{q'}}  \bra{\Omega} T \left(  c^{r'}_{q'} \left( \infty \right)  b_{p'}^{s'}  \left( \infty \right) \left( b^{s}_p \right)^{\dag} \left( -\infty \right) \left( c_{q}^{r} \right)^{\dag} \left( -\infty \right)    \right) 
.\end{align}

We want $\left( b_p^{s} \right)^{\dag} \left( \infty \right) - \left( b_p^{s} \right)^{\dag} \left( -\infty \right) $.

Consider
\begin{align}
    i \int \dd{^{4}x} \overline{\psi}\left( x \right) \left( i \overset{\leftharpoonup}{\fbs{\partial}} + m \right) u_s \left( p \right) e^{ipx}
.\end{align}

Recall that $u_s \left( p \right) $ obeys $\left( \fbs{p} - m \right) u_s \left( p \right) = 0$ and $\omega_p = \sqrt{\vec{p}^2 + m^2}$.

