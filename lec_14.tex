\lecture{14}{11/11/2024}{Feynman Rules}

\begin{note}
    $\Delta_{xy} = \bra{0} T \phi^{0}\left( x \right) \phi^{0}\left( y \right) \ket{0}$ and $\hat{\Delta}_{xy} = \bra{0} T \psi^{0} \left( x \right) \psi^{0\dag} \left( y \right) \ket{0}$
\end{note}

We now state the Feynman rules for scalar Yukawa theory (for the connected $S$-matrix).

\begin{enumerate}
    \item Draw all possible connected diagrams for the relevant process (i.e. fixed initial and final states).
    \item For each diagram the contribution to the $S$ matrix is:
        \begin{enumerate}[label=\roman*)]
            \item Assignment momenta to each line.
            \item Internal lines should be given by
                \begin{align}
 \feynmandiagram [baseline=(a.base), horizontal=a to b] {
     a -- [scalar,momentum'={$\vec{p}$}] b 
%b -- [boson,edge label=\(A^\mu\)] d 
 }; &= \int \frac{\dd{^4p}}{\left( 2\pi\right)^4} \frac{i}{p^2 - m^2 - i \epsilon}, \\
 \feynmandiagram [baseline=(a.base), horizontal=a to b] {
     a -- [fermion,momentum'={$\vec{k}$}] b 
%b -- [boson,edge label=\(A^\mu\)] d 
 }; &= \int \frac{\dd{^4k}}{\left( 2\pi\right)^4} \frac{i}{k^2 - m^2 - i \epsilon}, 
                .\end{align}
            \item For external lines do nothing.
            \item For each vertex, write the factor
                \begin{align}
                    \feynmandiagram [baseline=(a.base), horizontal=a to b] {
     a -- [scalar,momentum'={$\vec{p}_1$}] b -- [fermion,rmomentum'={$\vec{p}_2$}] c; 
     d --[fermion,rmomentum'={$\vec{p}_3$}] b;
%b -- [boson,edge label=\(A^\mu\)] d 
 }; = \left( -ig \right) \left( 2\pi \right)^{4} \delta \left( p_1 - p_2 + p_3 \right) 
                .\end{align}
            \item Sum over all diagrams and integrate over undetermined (loop) momenta.
        \end{enumerate}
\end{enumerate}

\begin{note}
    External particles are on shell as $p_i^2 = M^2$. However internal particles are not as $k^2 \neq m^2$. This gives them the name \textit{virtual particles}.

    What we computed is a tree-level diagram. This means we do not have any loops in our diagrams. Tree-level diagrams are the leading connected contributions and the emphasis of this course.
\end{note}

\begin{example}
    We move to study nucleon--anti nucleon scattering: $\psi^{\dag} \psi \to \psi^{\dag} \psi$ at leading order (+ connected).

    We have that
    \begin{align}
        \bra{f} S \ket{i}_c &=  \vcenter{\hbox{\feynmandiagram [vertical=a to b] {
    f --[fermion,momentum'={$\vec{p}_1$}] a -- [scalar,momentum'={$\vec{k}$}] b -- [fermion,rmomentum'={$\vec{p}_2$}] c; 
     d --[fermion,rmomentum'={$\vec{p}_3$}] b;
     g --[fermion,rmomentum={$\vec{p}_4$}] a;
%b -- [boson,edge label=\(A^\mu\)] d 
};}} + \vcenter{\hbox{\feynmandiagram [horizontal=a to b] {
    f --[fermion,momentum'={$\vec{p}_1$}] a -- [scalar,momentum'={$\vec{k}$}] b -- [fermion,momentum'={$\vec{p}_2$}] c; 
     d --[fermion,rmomentum'={$\vec{p}_3$}] b;
     g --[fermion,momentum={$\vec{p}_1$}] a;
};}} + \mathcal{O}\left( g^3 \right)  \\
&= \int \frac{\dd{^4k}}{\left( 2\pi\right)^4} \frac{i}{k^2 - m^2 - i\epsilon} \left( -ig \right) \left( 2\pi \right)^{4} \delta \left( p_1 - k - p_3 \right) \left( -ig \right) \left( 2\pi \right)^{4} \delta \left( k + p_2 - p_4 \right) \nonumber \\
&\quad + \int \frac{\dd{^4k}}{\left( 2\pi\right)^4} \frac{i}{k^2 - m^2 - i \epsilon} \left( -ig \right) \left( 2\pi \right)^{4} \delta \left( p_1 + p_2 + k \right) \left( -ig \right) \left( 2\pi \right)^{4} \delta \left( -k -p_3 - p_4 \right)  \\
&= \left( -ig \right)^2 \left( 2\pi \right)^{4} \delta \left(  p_1 + p_2 - p_3 - p_4\right) \left( \frac{i}{\left( p_1 - p_3 \right)^2 - m^2 - i \epsilon} + \frac{i}{\left( p_1 - p_2 \right)^2 - m^2 - i \epsilon} \right) + \mathcal{O}\left( g^3 \right)
    .\end{align}
    This second diagram has an interesting momentum dependence.

    In the center of mass frame,
    \begin{align}
        \left( p_1 + p_2 \right)^2 = 4 \left( M^2 + \vec{p}_1^2 \right) 
    .\end{align}
    Then
    \begin{align}
        \left( p_1 + p_2 \right)^2 - m^2 = 4 \left( M^2 + \vec{p}_1^2  \right) - m^2
    .\end{align}
    If $m < 2M$, we will never have $\left( p_1 + p_2 \right)^2 = m^2$ and thus can remove $i \epsilon$.
    If $m < 2M$, then for some $\vec{p}_1$ it is possible. This leads to a bump.
\end{example}

\subsection{Mandelstam Variables}

For $2 \to 2$ scattering the same combinations of momenta appear often. We define
\begin{align}
    s &= \left( p_1 + p_2 \right)^2 = \left( p_3 + p_4 \right)^2 \\
    t &= \left( p_1 - p_3 \right)^2 = \left( p_2 - p_4 \right)^2 \\
    u &= \left( p_1 - p_4 \right)^2 = \left( p_2 - p_3 \right)^2
.\end{align}

Where we assume $p_1 + p_2 = p_3 + p_4$.

With these variables we see our two diagrams were representing the $t$ and $s$ channel respectively.

For $\psi \psi \to \psi \psi$ scattering we have $t$ and $u$ channels respectively.

Assuming all particles have the same mass we have that given $p_1 = \left( E, p\hat{z} \right) $ and $p_2 = \left( E, - p \hat{z} \right) $, we have $p_3 = \left( E, \vec{p} \right) $ and $p_4 = \left( E, - \vec{p} \right) $ with $\vec{p} = \left( p \sin \theta, p \cos \theta, 0 \right) $ in the center of mass frame.

Notice that
\begin{align}
    s = 4E^2  &&
    t = -2p^2 \left( 1 - \cos \theta \right)  &&
    u = -2p^2 \left( 1 + \cos \theta \right) 
,\end{align}
which give us $s + t + u = 4m^2$. One can think of $s$ as the total energy of the system in the center of mass frame. $t$ and $u$ are measures of momentum exchange.





