\lecture{9}{30/10/2024}{Scattering}

One can picture the scattering of one state into another in different pictures,
\begin{align}
    \underbrace{\bra{\text{final;}~t_i} \ket{\text{initial;}~t_i}}_{\text{Schrödinger}}  = \underbrace{\bra{f} S \ket{i}}_{\text{Heisenberg}}
,\end{align}
one where the states evolve and one where the operators do.

We assume the Hamiltonian does time evolution such that
\begin{align}
    i \partial_t \phi = \left[ \phi, \mathcal{H} \right] 
,\end{align}
where $\phi$ can also be any operator in the theory.

We declare (assume) that at some time $t = t_0$, we can match the Hilbert space of $\mathcal{H}_0$ to that of $\mathcal{H}$ with
\begin{align}
    a_p \left( t \right) = e^{i H \left( t - t_0 \right) } a^{0}_p e^{-i H \left( t - t_0 \right) }
.\end{align}

Then for a field in the interacting theory, we can write
\begin{align}
    \phi_\text{int} \left( x \right) = \int \frac{\dd{^3p}}{\left( 2\pi \right)^3} \frac{1}{\sqrt{2 \omega_p} } \left( a_p \left( t \right) e^{-i p x} + a_p^{\dag} \left( t \right) e^{i p x} \right) 
.\end{align}

With this we can write down states as
\begin{align}
    \ket{\text{initial; }t_1} &= \ket{p_1 p_2} = \sqrt{2 \omega_1 }  \sqrt{2 \omega_2}  a^{\dag}_{p_1} a^{\dag}_{p_2} \ket{\Omega} \\
    \ket{\text{final; }t_2} &= \ket{p_3 p_4} = \sqrt{2 \omega_3 }  \sqrt{2 \omega_4}  a^{\dag}_{p_3} a^{\dag}_{p_4} \ket{\Omega}
.\end{align}

With the definition of asymptotic states, we will want the interactions to be turned off when $t_i \to - \infty$, $t_f \to \infty$. In this limit
\begin{align}
    \lim_{t \to \pm\infty}  a^{\dag}_p\left( t \right) = a_p^{0\dag}
.\end{align}

Naturally we need to figure out how to relate states at $\pm \infty$. We see that
\begin{align}
    \bra{f} S \ket{i} &= \bra{\text{final; }t_f} \ket{\text{initial; } t_i} \\
    &= \left( \prod_{i = 1}^{4} \sqrt{2 \omega_i}   \right) \bra{\Omega} T a_{p_3} \left( + \infty \right) a_{p_4} \left( + \infty \right) a_{p_1}^{\dag} \left( - \infty \right) a_{p_2}^{\dag} \left( - \infty \right) \ket{\Omega}
.\end{align}

\begin{claim}
    We claim
    \begin{align}
        \sqrt{2 \omega_p}  \left( a_p^{\dag} \left( \infty \right) - a^{\dag}_p \left( - \infty \right)  \right) = -i \int \dd{^{4}x} e^{-i p x} \left( \Box + m^2 \right) \phi \left( x \right) 
    .\end{align}
\end{claim}

\begin{proof}
    In the interacting theory, and here we have $\omega_p = \sqrt{\vec{p}^2 + m^2} $. We begin from the answer on the right. Observe that
    \begin{align}
        -i \int \dd{^{4}x} e^{-i px} \left( \Box + m^2 \right) \phi \left( x \right) &= -i \int \dd{^{4}x} e^{-i p x} \left( \partial_t^2 - \grad^2 + m^2 \right)\phi
    .\end{align}
    Using integration by parts, we see that
    \begin{align}
        -i \int \dd{^{4}x} e^{-i p x} \left( \partial_t^2 - \grad^2 + m^2 \right)\phi &= -i \int \dd{^{4}x} e^{-i px}\, ( \partial_t^2 + \underbrace{\vec{p}^2 + m^2}_{\omega_p^2} ) \,\phi \left( x \right) \\
       &= -i \int \dd{^{4}x} \partial_t \left( e^{-i p x} \partial_t \phi \left( x \right) - \left( \partial_t e^{-i p x} \right) \phi \left( x \right)  \right) 
    ,\end{align}
    which only depends on $t \to \pm \infty$.
    Recall from the free theory that
    \begin{align}
        \sqrt{2 \omega_p}  a^{0}_p &= i \int \dd{^3x} e^{ipx} \overset{\leftrightarrow}{\partial_t} \phi \left( x \right)  \\
        \sqrt{2 \omega_p}  a^{0\dagger}_p &= -i \int \dd{^3x} e^{-ipx} \overset{\leftrightarrow}{\partial_t} \phi \left( x \right)
    ,\end{align}
    where $f \overset{\leftrightarrow}{\partial_t} g = f \partial_t g - \left( \partial_t f \right) g$.
    This allows us to write
    \begin{align}
        \sqrt{2 \omega_p}  \int_{-\infty}^{\infty} \dd{t} \partial_t a_p^{\dag} \left( t \right) &= \sqrt{2 \omega_p} \left( a_p^{\dag} \left( \infty \right) - a_p^{\dag} \left( - \infty \right)  \right)
    ,\end{align}
    and analogously,
    \begin{align}
        \sqrt{2 \omega_p}  \left( a_p \left( \infty \right) - a_p \left( -\infty \right)  \right) = i \int \dd{^{4}x} e^{ipx} \left( \Box + m^2 \right) \phi \left( x \right) 
    .\end{align}

\end{proof}


With this, we can now write our desired inner product expression as
\begin{align}
    \bra{f} S \ket{i} 
    &= \left( \prod_{i = 1}^{4} \sqrt{2 \omega_i}   \right) \bra{\Omega} T \left( a_{p_3} \left( + \infty \right) - a_{p_3} \left( -\infty \right)  \right)  \left( a_{p_4} \left( + \infty \right) - a_{p_4} \left( -\infty \right)  \right)  \left( a_{p_1}^{\dag} \left( - \infty \right) - a_{p_1}^{\dag} \left( \infty \right)  \right)   \left( a_{p_2}^{\dag} \left( - \infty \right) - a_{p_2}^{\dag} \left( \infty \right)  \right)  \ket{\Omega} \\
    &= \prod_{j=1}^{4} \left( i \int \dd{^{4}x_j} \right)  \underbrace{e^{-i p_1 x_1} \left( \Box_1 + m^2 \right) e^{-i p_2 x_2} \left( \Box_2 + m^2 \right)}_{\text{ingoing}} \underbrace{e^{i p_3 x_3} \left( \Box_3 + m^2 \right) e^{ip_4 x_4} \left( \Box_4 + m^2 \right)}_{\text{outgoing}} \nonumber \\
    &\quad \times \underbrace{\bra{\Omega} T \phi \left( x_1 \right) \phi \left( x_2 \right) \phi \left( x_3 \right) \phi \left( x_4 \right) \ket{\Omega}}_{\text{4 point correlation function}}  
.\end{align}

This is the \textbf{LSZ reduction formula} for 2-2 scattering.


This has many advantages.
\begin{itemize}
    \item It is a manifestly Lorentz invariant $S$-matrix by construction (we don't even have to check).
    \item It makes clear the relation between $n$ point correlation functions $\bra{\Omega} T \phi \left( x_1 \right) \cdots \phi \left( x_n \right) \ket{\Omega}$ and $\bra{f} S \ket{i}$.
\end{itemize}


\begin{note}
    This process involved factoring out the operator
    \begin{align}
        \bra{\Omega} T \left( \Box_x + m^2 \right) \phi \left( x \right) \cdots \ket{\Omega} \to \left( \Box_x + m^2 \right) \bra{\Omega} T \phi \left( x \right) \cdots \ket{\Omega}
    ,\end{align}
    which is not strictly equal as there are contact terms unaccounted for. However these are not physically important as they are essentially the identity part of the $S$ matrix, $S \sim \mathbb{I} + i T$ which tells us that it is possible for things not to scatter at all. The \textbf{transfer matrix} $T$ is the interesting part of the $S$ matrix.
\end{note}

